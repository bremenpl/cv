\documentclass{tccv}
\usepackage[polish]{babel}	
\usepackage[utf8]{inputenc} % chars encoding
\usepackage[T1]{fontenc} % polish fonts
\usepackage[super]{nth} % 1st, 2nd etc
\usepackage[yyyymmdd]{datetime}

\begin{document}

\part{Łukasz Przeniosło}

\section{Work experience}

\begin{eventlist}

\item{July 2017 -- Present}
     {Przenioslo Electronics \& Software, Szczecin}
     {Hardware \& Software Engineer (Owner)}
     
Hardware and software design according to the client's needs and/ or specifications. See \hyperref[sec:clients]{Appendix} sections for more clients and projects references. 

\item{May 2020 -- December 2021}
     {Icotera Sp. z o.o., Szczecin}
     {Hardware Subject Matter Expert}
     
Hardware development processes management and verification.

\item{June 2013 -- July 2017}
     {Mechatronic Engineering Sp. z o.o., Szczecin}
     {Hardware \& software engineer}
     
Development and maintenance of hardware for the produced SMT machinery, writing firmware for the created hardware, writing testing PC applications, building prototypes. 

\item{July 2012 -- June 2013}
     {Mechatronic Engineering Sp. z o.o., Szczecin}
     {Hardware \& manufacture engineer}

SMT machinery hardware assembly, faulty parts service, existing designs debugging. 

\end{eventlist}

\section{Education}

\begin{yearlist}

\item[MA diploma]{2014--16}
     {Electrical engineering}
     {West Pomeranian University of Technology, Szczecin}

\item[BA diploma]{2010--14}
     {Electronics engineering}
     {West Pomeranian University of Technology, Szczecin}

\end{yearlist}

\section{Courses \& licenses}

\begin{yearlist}

\item{2022}
     {Sages MISRA C}
     {Safe code based on MISRA C course}
     
\item{2020}
     {The Technology Academy}
     {RF \& Microwave course}

\item{2018}
     {C++ Institute CLA \& CLP}
     {Advanced C11 programming course}

\item{2017}
     {Unmanned Aerial Vehicle Operator}
     {Visual Line of sight (VLOS) license}

\end{yearlist}

\vspace*{\fill} % put the next formula at the bottom of the page

{\scriptsize Document last update date and time: \today, \currenttime. Up-to-date CV always under  \href{https://github.com/bremenpl/cv/blob/master/tex/cv.pdf}{github.com/bremenpl/cv/blob/master/tex/cv.pdf}}

\personal
    [www.przenioslo.com]
    {Goleniów, Poland}
    {+48 792 456 829}
    {lukasz@przenioslo.com}
    
\section{Honors and scholarships}

\begin{yearlist}

\item{2017}
     {Szczecin's city president best thesis award}
     {Received for MA thesis: \href{https://github.com/bremenpl/praca_mgr}{Universal smart electric motors controller for industry applications}} 
     
\item{2015}
     {Polish Minister of science and higher education scholarship}
     {Received in 2015 for academic achievements} 
    
\end{yearlist}

\section{Skills}

\begin{factlist}

\item{HW}
     {PCB design \& production, soldering, rapid prototyping, Altium Designer, Orcad/ Allegro, AVR, PIC, ARM, PowerPC, STM32, FPGA, C2000, MEMS, RF/analog, power electronics, low power, BMS, SBC}

\item{SW}
     {Assembly, C, Modern C++, MISRA C, Qt, QML, Matlab, Plecs, Spice, GIT, VHDL, DSP, FreeRTOS, motor control, Buildroot/ Yocto (Poky), Linux drivers, IoT, OPCUA, Unit testing}
     
\item{MISC}
     {Documentation (\LaTeX, Doxygen, Office), HW/SW product management (design, pricing, production, BOM optimization)}

\end{factlist}

\section{Communication}

\begin{factlist}
\item{Polish}{Native language}
\item{English}{Full professional proficiency}
\item{German}{Elementary proficiency}
\end{factlist}

\section{Social networking}

\begin{factlist}

\item{Linkedin}
     {\href{http://pl.linkedin.com/in/przenioslo}{linkedin.com/in/przenioslo}}
     
\item{Github}
     {\href{http://github.com/bremenpl}{github.com/bremenpl}}
     
\item{Github}
     {\href{http://github.com/przenioslo}{github.com/przenioslo}}

\end{factlist} 

\vspace*{\fill} % put the next formula at the bottom of the page

{\tiny I agree to the processing of personal data provided in this document for realising the recruitment process pursuant to the Personal Data Protection Act of 10 May 2018 (Journal of Laws 2018, item 1000) and in agreement with Regulation (EU) 2016/679 of the European Parliament and of the Council of 27 April 2016 on the protection of natural persons with regard to the processing of personal data and on the free movement of such data, and repealing Directive 95/46/EC (General Data Protection Regulation)}

\clearpage

\onecolumn

\section{Appendix A: extended competence list (Hardware)}

\begin{itemize}
	\item Digital circuits development based on discrete components or advanced IC's
		\begin{itemize}
			\item MCU based designs, using IC's such as ARM Cortex M0/M3/M4/M7, PIC, AVR or C2000 families,
			\item CPU based designs, using IC's such as NXP i.MX 6 and i.MX 7 series application processors or TI AM335x Sitara application processors,
			\item FPGA based designs, using IC's such as Xilinx Spartan family or Lattice Mach family,
			\item experience in high speed designs for SBCs (Single Board Computers) consisting of memories IC's such as: NOR/ NAND Flash, DDR3 RAM, SD cards and eMMC chips, Sata drives,
			\item familiar with high speed designs utilized for: 
				\begin{itemize}
				\item reducing cross-talk and distortions, 
				\item reducing ground bounce, 
				\item reducing radiation (EMI), 
				\item differential pairs design and routing,
				\end{itemize}
			\item familiar with signaling/equalization and signal integrity provision techniques
			\item knowledge about serial interfaces, such as: UART/ USART, I2C, I2S, SPI, QSPI, CAN, LIN, Ethernet, Ethercat.
		\end{itemize}
\end{itemize}

\begin{itemize}
	\item Analog circuits development based on discrete components and dedicated IC's
	\begin{itemize}
		\item experience in audio analog front-end designs,
		\item experience in building low voltage measurement circuits,
		\item utilizing high resolution ADC's and DAC's,
		\item good knowledge about analog circuits shielding and separation (i.e. ground start connections, guard rings, via shielding and stitching),
		\item experience in RF analog front-end design. Utilized techniques: 
		\begin{itemize}
			\item output to antenna matching impedance circuits (Wavelength and Microstrip),
			\item PCB antenna length tuning,
			\item knowledge about network analyzer usage,
		\end{itemize}		 
		\item built devices in the following technologies/ frequencies (and wrote firmware for them):
		\begin{itemize}
			\item Wifi, 2.4 Ghz (ESP8266 and ESP32),
			\item Zigbee, 2.4 Ghz and 868 Mhz (Digi Xbee Digimesh),
			\item Bluetooth, 2.4 Ghz (Nordic NRF chips),
			\item 2G/ 3G, ~820 - 2200 Mhz (u-Blox SARA U201), 
		\end{itemize}
		\item good knowledge about the principles of operation of the basic discrete components such as BJT's, Mosfets, OP Amp's, Flip-flops, Multiplexers/ Demultiplexers etc.
	\end{itemize}
\end{itemize}

\begin{itemize}
	\item Power electronics circuits development based on discrete components and dedicated IC's
	\begin{itemize}
		\item motor control drivers (from ground-up) for the following motor types: Stepper motors, DC motors, VCM (Voice coil) motors, BLDC and PMSM,
		\item experience in creating hardware for industry grade robots utilized in SMT production, such as: Pick and Place machines, Stencil printers, conveyors and reflow ovens,
	\item DC-DC converters (Buck and Boost converters),
	\item built and programmed high voltage/ current (Lithium and Lead acid based) Battery Management Systems (BMS) and Uninterruptible power supplies (UPS) for power backup applications (civil, maritime and medical),
	\item experience in building various battery chemistry devices and chargers (i.e.: Li-ion, Li-pol, LiFePo4 (LFP), Nimh, lead acid), 
	\item experience in building low power, energy harvesting, battery powered IoT devices.
	\end{itemize}
\end{itemize}

\begin{itemize}
	\item CAD/ CAM/ Simulation technologies, tools knowledge and usage experience:
	\begin{itemize}
		\item long time Altium Designer user,
		\item experience in schematics design and simulation,
		\item experience in mixed signals design's PCB creation consisting of up to 12 layers stackups,
		\item experience in creating multi PCB designs
		\item experience in using SPICE and other simulation tools, such as: LTSpice, SIMetrix, Simulink, Plecs,
		\item experience in WiFi technology based (incl. AX) hardware development and testing.
		
	\end{itemize}
\end{itemize}
     
\clearpage

\section{Appendix B: extended competence list (Software)}

\begin{itemize}
	\item Hardware description languages:
	\begin{itemize}
		\item proficient in VHDL code design,
		\item less experienced in Verilog code design. 
	\end{itemize}
	\item Assembly
		\begin{itemize}
			\item experienced in AVR Assembler space efficient code development for memory constrained devices,
			\item generic knowledge of ARM and x86 assemblers for debugging purposes.
		\end{itemize}
	\item C
		\begin{itemize}
			\item long time experience in C89, C99 and C11 standars usage,
			\item bare metal applications (no operating system),
			\item real time operating systems applications, such as FreeRTOS,
			\item embedded Linux based applications (ARM and PowerPC),
			\item high efficiency x86 multiplatform applications,
			\item Linux Kernel drivers development (character and network),
			\item knowledge in the memory management field (MMU, DMA, dynamic memory allocation, memory structure architectures),
			\item experience in multiprocess and multithread applications (good knowledge of multithreading principles),
			\item experience in using generic and self written DSP libraries for applications such as: PID control, Fuzzy Logic control, audio signals processing, measurement data processing).
			\item experience in using MISRA C and various Linter applications.
			\item long time experience in designing firmware for various power management devices (BMS, UPS, motor control, switched converters).
			\item long time experience in various Bootloader programs design.
		\end{itemize}
	\item C++
		\begin{itemize}
			\item long time experience in C++11, C++14 and C++17 standards,
			\item build efficient applications for multiple operating systems: Windows, Linux, Mac OS, iOS, Android,
			\item utilizing modern C++ concepts, such as Smart Pointers, Futures, Lambdas, Templates, Move semantics,
			\item familiar with design patterns and principles such as SOLID or RAII,
			\item familiar with Unit Testing principles,
			\item experience in multithreaded application in low and high level domain,
			\item long time experience in using Qt with QML and/ or Felgo frameworks. Utilized Qt technology for building truly multiplatform (desktop and mobile) applications,
			\item experience in creating event driven applications,
			\item built both backend (headless) and front end (GUI) applications,
			\item knowledge about maintaining good balance between code readability/ quality and high performance,
			\item knowledge of data structures and algorithms,
			\item experience in both low level (TCP/IP, UDP) and high level (HTTP, FTP, SFTP, OPCUA, MQTT etc.) networking protocols and applications.
		\end{itemize}
	\item Tools and Operating Systems
		\begin{itemize}
			\item proficient in Unix/Linux, Windows and MacOs environments,
			\item Linux build systems maintenance and design using Buildroot and Yocto (Poky) tools,
			\item familiar with make, qmake and cmake building tools,
			\item worked with multiple compilers: MSVC, GCC and LLVM,
			\item working efficiently with GIT version control (and SVN if forced to),
			\item familiar with Valgrind dynamic analysis tool,
			\item familiar with GDB debugging tool both locally and remotely,		
			\item familiar with Gtest, Ceedling and Catch2 unit testing frameworks, 	
			\item familiar with Jira and Confluence management and documentations tools.		
			
		\end{itemize}
\end{itemize}

\clearpage

\section{Appendix C: extended competence list (Miscellaneous)}

\begin{itemize}
	\item Proficient in documentation preparation using 
	\begin{itemize}
		\item \LaTeX, 
		\item Doxygen,
		\item MS Office/ Libre Office
		\item Confluence
	\end{itemize}
	\item experience in hardware, software and mixed type of products leading in small teams. Long time interdisciplinary experience provides good diversity for various projects,
	\item can act as a standalone developer or a team player in a project,
	\item good at multitasking, can handle multiple sub-tasks simultaneously,
	\item experienced with developer to client relations handling,
	\item experienced with client to ODM relations handling,
	\item good at Power Point presentations (both preparing and giving them),
	\item experienced with working in multicultural environments.
	\item experienced with leading engineering teams in various embedded projects.
	\item experienced in 3D design (Rhino) and general 3D printing (mostly for support products, such as electronics devices prototype cases).
	\item long time Beaglebone and Raspberry Pi based systems designer (HW and FW/ SW).
	 
\end{itemize}

\clearpage

\section{Appendix D: Work references (clients/ employers)}
\label{sec:clients}

This section lists some of the companies (clients or employers) I have worked for in the past. It provides only the basic information that is not covered by the NDA's/ contracts, such as company name, manager/ supervisor name + position at the time and contact info. The list is not sorted in any specific way- it is (time wise) randomly ordered. \\

\begin{yearlist}

\item[Kim Esben Jorgensen (CTO), Danny van der Poel (CCO)]{•}
     {Icotera Sp. z o.o.}
     {contacts: \href{https://www.linkedin.com/in/kim-esben-j\%C3\%B8rgensen-8685229/}{Linkedin}, \href{https://www.linkedin.com/in/dannyvanderpoel/}{Linkedin}}

\item[Zygmunt Mijakowski (R\&D director)]{•}
     {Mechatronic Engineering Sp. z o.o.}
     {contact: \href{https://www.linkedin.com/in/zmijakowski/}{Linkedin}}
     
\item[Joerg Hering (Chief of Hardware \& Software dept)]{•}
     {MacGregor Germany GmbH \& Co. KG}
     {contact: \href{mailto:jhe@danelec.com}{email}}
     
\item[Peer Mork (CEO), Morten Ford (Head of Projects \& International Defense)]{•}
     {idoc A/S}
     {contacts: \href{https://www.linkedin.com/in/peermork}{Linkedin}, \href{https://www.linkedin.com/in/morten-ford-08368264}{Linkedin}}
     
\item[Krzysztof Penkala (PhD, EEng, Senior Lecturer)]{•}
     {West Pomeranian University of Technology}
     {contact: \href{https://www.linkedin.com/in/krzysztof-penkala-4a498a26/}{Linkedin}}
     
\item[Vinit Dipak Gandhi (Head of Business Development)]{•}
     {Global Power Source Pte. Ltd. }
     {contact: \href{https://www.linkedin.com/in/vinit-dipak-gandhi-71782612}{Linkedin}}
     
\item[Kamil Lata (Project manager]{•}
     {LINA Medical Polska Sp. z o.o.}
     {contact: \href{https://www.linkedin.com/in/kamil-lata-a71a41138}{Linkedin}}
     
\item[Robert Rak (CCO)]{•}
     {Apptimia Sp. z o.o.}
     {contact: \href{https://www.linkedin.com/in/robert-rak-4018a64a}{Linkedin}}

\item[Michał Bonisławski (CCO)]{•}
     {Mpower Sp. z o.o.}
     {contact: \href{https://www.linkedin.com/in/micha\%C5\%82-bonis\%C5\%82awski-322bb91aa}{Linkedin}}

\end{yearlist}

\clearpage

\section{Appendix E: Work references (projects)}

The section presents some of the projects I have participated in commercially. In some of them I participated partially, as an engineer in a bigger team. In others, I participated fully, providing a complete product from ground-up. The order in which the projects are listed is random and has no correlation with the order of listed companies  \hyperref[sec:clients]{in the prior section}. \\

\begin{yearlist}

\item[Full product bringup]{•}
     {Battery management system (BMS) based on lead acid and lithium chemistry cells}
     {Designed the complete product from ground-up, including: client requirements evaluation, hardware components proposals, hardware schematics and PCB design, hardware simulations, firmware development, BOM optimization, prototype and mass production management and scheduling. The end product was intended for high power battery management installations in critical infrastructures, such as hospitals or server farms. The designed BMS allowed for accurate telemetry of the battery electrical and physical parameters. Self developed algorithm allowed for accurate battery internal impedance (IR) calculations.}
     
\item[Hardware description and kernel drivers development]{•}
     {FPGA based SBC controller module}
     {Designed a board controller for a client provided, existing single board computer. The work involved complete VHDL code design for the controller FPGA (managing different hardware parts in real time), as well as writing the Linux kernel driver for the main on-board SoC communicating with the given controller.}
     
\item[Full product bringup]{•}
     {Integrated impedance scanner for biomeasurement applications}
     {Designed the hardware, firmware and software for a medical device prototype used to measure electrical parameters, allowing to perform accurate bioimpedance calculations.}
     
\item[Hardware and firmware development]{•}
     {IoT self-sustainable, smart weather stations}
     {Designed the hardware and firmware for networked weather stations used for crops harvesting process optimization. The design involved low power design in both hardware and firmware layers. Multiple stations were placed on a single crops field to periodically measure atmospheric parameters, which would then be fed to the centralized server via GSM technology. Server backend and web frontend were developed by other team members.}
     
\item[Hardware and firmware development]{•}
     {Power line communication (PLC) emergency lights system}
     {Designed the firmware, software and parts of the hardware for an industrial scale system used for emergency lights power cycling management. The communication with smart light modules was done over PLC, which allowed for feasible integration within legacy electrical grids in old buildings.}   
     
\item[Hardware and firmware development]{•}
     {Semi-automatic stencil printer}
     {Designed the hardware, firmware and PC control software for a semi-automatic Stencil Printer (device used for solder paste distribution over PCBs during production). The design involved real time I/O handling and motor control.}
     
\item[Software development]{•}
     {Seamanship improvement voyage suite}
     {Created the backend and frontend for a seamanship improvement software to be used in the maritime market. The application used in different modes, could be utilized by the captains on the ship's bridge, as well as by the operators in fleet operation centers. The aim of the project was to reduce potential nautical incidents during cruising.}        
     
\item[Project management and hardware design consulting]{•}
     {High efficiency wireless + wired gateway}
     {Overseen the hardware product development, i.e. validated and proposed electronic design and thermal solutions, created hardware simulation scenarios, optimized bill of materials (BOMs), conducted hardware tests, designed prototypes.}             

\item[Firmware and software development]{•}
     {SMD components automatic feeder}
     {Designed the firmware for various automatic feeder devices used with SMT Pick \& Place machines. Created the PC suite software used for manufacturing and testing purposes. Created and assembled prototypes.}

\end{yearlist}

\begin{yearlist}
     
\item[Firmware development]{•}
     {Automatic feeder controller}
     {Designed the firmware for a device used for bridging multiple automatic feeders (connected to the feeders rack) with the main logic unit in a Pick \& Place machine. The board allowed for hardware interfaces and serial protocol interfaces matching/ translation between different system parts.}
     
\item[Hardware and firmware development]{•}
     {Smart UPS}
     {Designed the hardware and firmware for a multi-functional UPS device intended for the maritime market. It allowed for the configuration of most of the electrical, operational parameters (over multiple USB endpoints/ protocols). It consisted of multiple electrical protections (implemented in hardware in firmware) and allowed for smart charging and discharging of the cells to optimize energy utilization.}
     
\item[Hardware and firmware development]{•}
     {Soldering paste dispenser head}
     {Designed the hardware and firmware for a multi-functional/ multi-modal soldering paste dispenser head mounted in a Pick \& Place machine. It allowed for efficient soldering paste distribution over different subtracts (mostly PCBs), utilizing different dispensing methods (i.e., solder paste squeezing or "spitting"). It kept the mounted solder paste within a controlled temperature to avoid drying.}
     
\item[Firmware development]{•}
     {Electrosurgery knife}
     {Designed the firmware for an electrosurgery device within a team of developers. The device was intended for a medical market and allowed for cutting and coagulating living tissues. Lead the firmware architectural design for the product. Conducted code reviews. Designed unit tests.}     
     
\item[Full product bringup]{•}
     {Multimotor controller}
     {Designed a complete device (hardware, firmware and PC tuning software) for dual (various) motor control in real time. The product supported two different motor types (DC, BLDC, PMSM, Stepper  or VCM) driving simultaneously. The devices was used as part of different industrial machines that had to utilize motor controlled linear or rotary movement. It consisted of numerous DPS algorithms implemented for smooth control operation.}  
     
\item[Full product bringup]{•}
     {\href{https://github.com/bremenpl/BeagleNodeHW}{Beaglenode}}
     {Authoring an open source/ open hardware project based on BeagleBone, which is under continuous development. It's aim is to allow for seamless lights and shutters control implementation in 110/ 230 V AC "star" shaped electrical home installations. The first hardware prototype was designed in Altium Designer. On the software side the product is being developed using the following technologies: Yocto, Qt, C++, MQTT.} 

\item[Hardware and firmware development]{•}
     {Production line conveyor}
     {Designed the hardware and firmware for a multi-segment conveyor utilized in automotive parts production industry. The product allowed for transporting the required parts between different automatic assembly stages on the production line. It consisted of functionalities such as parts scanning and "traffic" mitigation.}  

\end{yearlist}

\end{document}























